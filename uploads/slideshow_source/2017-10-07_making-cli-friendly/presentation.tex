\documentclass[]{beamer}
\usetheme{Antibes}
\usecolortheme{seagull}
\usepackage[utf8]{inputenc}
\usepackage{graphicx}
\usepackage{grffile}
\usepackage{longtable}
\usepackage{wrapfig}
\usepackage{rotating}
\usepackage[normalem]{ulem}
\usepackage{amsmath}
\usepackage{textcomp}
\usepackage{amssymb}
\usepackage{capt-of}
\usepackage{hyperref}

% colors
% \usepackage[dvipsnames]{xcolor}

% include external content
\usepackage{fancyvrb}
\usepackage{relsize}

% Code formatting
\newcommand{\cli}[1]{\texttt{#1}}
\usepackage{listings}
\lstset{%
  breaklines=true,
  linewidth=\textwidth,
  captionpos=b, % place captions below
}
\lstnewenvironment{sh}{\lstset{language=sh}}{}
\renewcommand{\sh}[1]{\lstinline[columns=fixed,language=sh]{#1}}

% Metadata
\newcommand{\mytitle}{Making the Command-Line Friendly}
\title{\mytitle{}}
\author{Joe LaFreniere (lafrenierejm)}
\institute{Linux Users Group @ UT Dallas}
\date{2017-10-07}
\hypersetup{
  pdfauthor={Joe LaFreniere (lafrenierejm)},
  pdftitle=\mytitle{},
  pdfkeywords={},
  pdfsubject={},
  pdflang={English}}
\begin{document}

\maketitle
\begin{frame}{Outline}
  \tableofcontents
\end{frame}

\section{Requirements}
\subsection{Story}
\begin{frame}
  \frametitle{User Story}
  \begin{enumerate}
  \item<1|only@1>
    Boot our Linux machine.
  \item
    Login and start a window manager.
  \item
    Open shell 1 in terminal 1 and SSH into server 1.
  \item
    Open shell 2 in terminal 2 and SSH into server 1.
  \item
    Open a shell 3 in terminal 2 and SSH into server 2.
  \item
    Close the SSH connection in shell 1 and SSH into server 2.
  \end{enumerate}
\end{frame}

\begin{frame}
  \frametitle{Artifacts}
  \begin{itemize}
  \item
    multiple terminal emulators
  \item
    multiple shells within each terminal, \(x\) total
    \begin{itemize}
    \item
      may be initialized before or after first connection
    \end{itemize}
  \item
    one or more SSH connection per shell, \(y\) total
  \item
    one or more remote servers, \(z\) total
  \end{itemize}
\end{frame}

\subsection{Security}
\begin{frame}
  \frametitle{Securing Keys}
  \begin{columns}
    \column{0.5\textwidth}<1->
    \begin{block}{Problem}
      \begin{enumerate}
      \item<2->
        secrets must remain secret
      \item<4->
        secrecy during use
      \item<6->
        encryption key per secret
      \item<8->
        brute-force passphrase
      \item<10->
        remember passphrase
      \item<12->
        one passphrase per key
      \item<14->
        plaintext is insecure
      \end{enumerate}
    \end{block}

    \column{0.5\textwidth}
    \begin{block}{Solution}<1->
      \begin{enumerate}
      \item<3->
        encrypt device storage
      \item<5->
        per-key encryption
      \item<7->
        per-key passphrase
      \item<9->
        long passphrase
      \item<11->
        password manager
      \item<13->
        cache plaintext keys
      \item<15->
        only cache keys in use
      \end{enumerate}
    \end{block}
  \end{columns}
\end{frame}

\begin{frame}
  \frametitle{Key Cache}
  \begin{enumerate}
  \item\pause
    allow for single unlock per key
    \begin{itemize}
    \item
      shells --- terminal emulators, TTYs
    \item
      graphical applications
    \end{itemize}
  \item\pause
    cache keys only as they are used
  \item\pause
    be stateless WRT initialization order
  \item\pause
    provide mechanism for clearing cache
    \begin{itemize}
    \item
      manually
    \item
      on timeout from cache
    \item
      on timeout from last use
    \end{itemize}
  \item\pause
    include hooks for password manager
  \end{enumerate}
\end{frame}

\section{Key Cache}
\subsection{\sh{ssh-agent}}
\begin{frame}
  \frametitle{\sh{ssh-agent}}
  \begin{itemize}
  \item\pause
    packaged with modern releases of OpenSSH
  \item\pause
    keys only accessible from owning user
    \begin{itemize}
    \item
      vulnerable to malicious root
    \end{itemize}
  \item\pause
    uses environment variables
  \item\pause
    separate instance per shell
  \item\pause
    not convenient for other applications
  \end{itemize}
\end{frame}

\begin{frame}
  \frametitle{Running \sh{agent}}
  \begin{block}{interactive shell}
    \lstinputlisting[%
    language=sh,
    label=lst:zshrc,
    tabsize=4,
    ]{./listings/zsh-agent-interactive.zsh}
  \end{block}
  \begin{block}{Problems}\pause
    It's dumb
  \end{block}
\end{frame}

\begin{frame}
  \frametitle{Running \sh{agent} from RC}
  \begin{block}{\sh{\$HOME/.zshrc}}
    \lstinputlisting[%
    language=sh,
    label=lst:zshrc,
    tabsize=4,
    ]{./listings/zshrc-agent-naive.zsh}
  \end{block}
  \begin{block}{Problems}\pause
    \begin{itemize}
    \item
      separate agent for each shell
    \item
      keys only cached within that session
    \end{itemize}
  \end{block}
\end{frame}

\subsection{\sh{keychain}}
\begin{frame}
  \frametitle{\sh{keychain}} 
  \begin{itemize}
  \item
    front-end to \sh{ssh-agent}
  \item
    long-running process rather than per-login
  \item
    list secret keys in configuration
  \item
    unlock all keys upfront
  \end{itemize}
  \vspace{2em}
  
  homepage: \href{http://www.funtoo.org/Keychain}{funtoo.org/Keychain}
  
  git repo: \href{https://github.com/funtoo/keychain}{github.com/funtoo/keychain}
\end{frame}

\begin{frame}
  \frametitle{Running \sh{keychain}}
  \begin{block}{\sh{\$HOME/.zshrc} (broken by comments)}<1|only@1>
    \lstinputlisting[%
    language=sh,
    label=lst:zshrc,
    tabsize=4,
    ]{./listings/zshrc-keychain-commented.zsh}
  \end{block}
  \begin{block}{\sh{\$HOME/.zshrc}}<2|only@2>
    \lstinputlisting[%
    language=sh,
    label=lst:zshrc,
    tabsize=4,
    ]{./listings/zshrc-keychain-working.zsh}
  \end{block}
\end{frame}

\subsection{\sh{ssh-agent}}
\begin{frame}
  \frametitle{Back to raw \sh{agent}}
  \begin{block}{\sh{\$HOME/.zshrc}}<1|only@1>
    \fontsize{8}{0}
    \selectfont{}
    \lstinputlisting[%
    language=sh,
    label=lst:zshrc,
    tabsize=4,
    ]{./listings/zshrc-agent-complete.zsh}
  \end{block}
  \begin{block}{Remaining Problems}<2|only@2>
    \begin{itemize}
    \item
      all keys in one agent
      \begin{itemize}
      \item
        forwarding that agent exposes all keys
      \end{itemize}
    \item
      requires cleanup of \sh{\$SSH_AUTH_SOCK} and \sh{\\tmp}
    \end{itemize}
  \end{block}
\end{frame}

\subsection{\sh{ssh-ident}}
\begin{frame}
  \frametitle{To-Do}
  \href{https://github.com/ccontavalli/ssh-ident}{github/ccontavlli/ssh-ident} claims to solve problems by storing each key in its own agent.
  \begin{itemize}
  \item
    written in Python
  \item
    open PR for BASH autocompletion
  \item
    requires replacing \sh{ssh} binary for dependent applications
  \end{itemize}
\end{frame}

\section{Requirements}
\subsection{Accessibility}
\begin{frame}
  \frametitle{Managing Key Passphrases}
  \begin{enumerate}
  \item<1->
    generation
    \begin{itemize}
    \item
      length
    \item
      symbols
    \end{itemize}
  \item<1->
    encryption
  \item<1->
    synchronization across devices
  \item<2->
    command-line interface
    \begin{itemize}
    \item
      access from other scripts/apps
    \item
      autocompletion from shell
    \end{itemize}
  \end{enumerate}
\end{frame}

\section{Password Manager}
\subsection{\sh{pass}}
\begin{frame}
  \frametitle{\sh{pass} (passwordstore.org)}
  ``The standard UNIX password manager''
  \vspace{1em}
  
  \begin{itemize}
  \item
    generation runs \sh{tr} on \cli{/dev/urandom}
    \begin{itemize}
    \item
      user sets length, symbols
    \end{itemize}
  \item
    git for synchronization
  \item
    CLI-first
    \begin{itemize}
    \item
      autocompletion for zsh, bash, fish
    \end{itemize}
  \end{itemize}
  \vspace{1em}
  homepage: \href{http://www.passwordstore.org}{passwordstore.org}
  
  git repo: \href{https://git.xz2c4.com/password-store/}{git.xz2c4.com/password-store}
\end{frame}

\section{Requirements}
\subsection{Usability}
\begin{frame}
  \frametitle{Shell Autocompletion}
  \begin{columns}
    \column{0.5\textwidth}<1->
    \begin{block}{General}
      \begin{enumerate}
      \item
        executables from \sh{\$PATH}
      \item
        arguments
        \begin{itemize}
        \item
          limit paths by type (\sh{find -type})
        \end{itemize}
      \item
        executables' options
      \item
        option arguments
      \end{enumerate}
    \end{block}
    \column{0.5\textwidth}<1|only@1>

    \column{0.5\textwidth}<2|only@2>
    \begin{block}{Password Manager}
      \begin{enumerate}
      \item
        arguments
        \begin{itemize}
        \item
          generate secret
        \item
          find filename
        \item
          copy secret
        \end{itemize}
      \item
        directories, subdirectories
      \item
        filename
      \end{enumerate}
    \end{block}

    \column{0.5\textwidth}<3|only@3>
    \begin{block}{\sh{SSH}, \sh{SCP}, \sh{rsync}\ldots{}}
      \begin{enumerate}
      \item
        hostname or IP address
      \item
        remote path
        \begin{itemize}
        \item
          prompt if key not cached
        \end{itemize}
      \item
        local path
      \end{enumerate}
    \end{block}
  \end{columns}
\end{frame}

\section{zsh Autocompletion}
\subsection{Writing}
\begin{frame}
  \frametitle{Getting Started}
  \begin{itemize}
  \item
    primary source is \sh{man zshcompsys}
    \begin{itemize}
    \item
       ``zsh completion system''
    \end{itemize}
  \item
    completions loaded from \sh{\$fpath}
  \item
    filename must start with an underscore
  \end{itemize}
  \begin{enumerate}
  \item
    create a directory to house custom/prototype completions
  \item
    add that directory to \sh{\$fpath} from \cli{.zshrc}
  \item
    initialize completion in \cli{.zshrc}
  \end{enumerate}
\end{frame}

\subsection{Using}
\begin{frame}
  \frametitle{Using Autocompletions}
  \begin{block}{\sh{\$HOME/.zshrc}}
    \lstinputlisting[%
    language=sh,
    label=lst:zshrc,
    tabsize=4,
    ]{./listings/zshrc-autocompletion.zsh}
  \end{block}
\end{frame}
\end{document}

%%% Local Variables:
%%% mode: latex
%%% TeX-master: t
%%% End:
